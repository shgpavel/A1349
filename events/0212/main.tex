%!TeX program = xelatex

\documentclass[aspectratio=169]{beamer}
\usepackage[main=russian,english]{babel}

\usepackage{fontspec}
\usepackage{graphicx}
\usepackage{xcolor}
\usepackage{xltxtra}
\usepackage{hyperref}
\usepackage{multicol}
\usepackage{unicode-math}
\usepackage{changepage}
\usepackage{caption}
\captionsetup[figure]{name=Fig.,labelsep=colon}
\setbeamertemplate{caption}[numbered]

\usepackage{tcolorbox}
\tcbuselibrary{skins}

\setmainfont{Iosevka NFM}
\setmonofont{Iosevka NFM}
\setsansfont{Iosevka NFM}
\setromanfont{Iosevka NFM Italic}
%\setmathfont{DejaVu Math TeX Gyre}
\setmathfont{Latin Modern Math}

\title{Планировщик процессов для гетерогенных процессорных архитектур}
\author{Шаго Павел Евгеньевич}
\institute {
  Научный руководитель: Корхов Владимир Владиславович \\
  \vspace{0.5cm}
  Факультет прикладной математики -- процессов управления\\
  Кафедра моделирования электромеханических и компьютерных систем
}

\usetheme{Warsaw}

\newtcolorbox{shadowbox}{
  enhanced,
  colback=white,
  colframe=white,
  boxrule=0pt,
  sharp corners=all,
  arc=6pt,
  drop shadow,
  left=4mm,
  right=4mm,
  top=2mm,
  bottom=2mm
}

\begin{document}

\begin{frame}
  \titlepage
\end{frame}

\begin{frame}
  \frametitle{Цель и задачи}

  \begin{block}{Цель}
    Разработать более эффективный планировщик процессов
    для гетерогенного процессора (Hybrid CPU, big.LITTLE)
  \end{block}

  \begin{block}{Задачи}
    \begin{enumerate}
      \item Реализовать алгоритм EEVDF с использованием sched\_ext
      \item Определить переменные, параметры и ограничения, построить модель
      \item Придумать и реализовать итоговый алгоритм используя sched\_ext
    \end{enumerate}
  \end{block}
\end{frame}

\begin{frame}
  \frametitle{sched\_ext}
  \begin{block}{Описание}
    Инфраструктура в ядре Linux для разработки планировщиков на eBPF
    и их исполнения в виртуальной машине ядра
  \end{block}

  \begin{block}{Особенности}
    \begin{itemize}
      \item Предоставляет полноценный API для реализации любого
        алгоритма планирования
      \item Динамическое подключение и отключение
      \item Безопасность: откат на EEVDF при ошибках
    \end{itemize}
  \end{block}

  \textbf{Цикл планирования:}
  \begin{enumerate}
    \item \texttt{select\_cpu()} $\to$ выбор потока процессора
    \item \texttt{enqueue()} $\to$ вставка в глобальную очередь DSQ
    \item \texttt{dispatch()} $\to$ перемещение в локальную DSQ
    \item \texttt{dequeue()} $\to$ извлечение из локальной очереди
  \end{enumerate}

\end{frame}

{
  \setbeamertemplate{footline}{}
  \setbeamertemplate{navigation symbols}{}

  \begin{frame}
    \begin{figure}[ht]
      \centering
      \includegraphics[width=1\linewidth]{res/libbpf.png}
      \caption{Схема работы eBPF}
      \label{fig:my_label}
    \end{figure}
  \end{frame}
}

{
  \setbeamertemplate{footline}{}
  \setbeamertemplate{navigation symbols}{}

  \begin{frame}
    \begin{figure}[ht]
      \centering
      \includegraphics[width=0.96\linewidth]{res/intel_arl.jpg}
      \caption{Гетерогенный процессор Intel (целевая платформа)}
      \label{fig:my_label2}
    \end{figure}
  \end{frame}
}

\begin{frame}
  \frametitle{Введение}

  \begin{itemize}
    \item Алгоритм построен на концепции сведения параметров задачи
      в одну функцию -- виртуальное время.
    \item Решение о планировании принимается так: выбирается процесс
      у которого есть доступный запрос с самым ранним
      виртуальным дедлайном.
    \item Главный результат: гарантия справедливости алгоритма
      EEVDF, задержка строго ограничена размером временного
      кванта $ q $.
    \item Показано, что ограничение является оптимальным для алгоритмов
      пропорционального распределения и улучшает все предыдущие ограничения.
  \end{itemize}

  \begin{shadowbox}
    Earliest Eligible Virtual Deadline First : A Flexible
    and Accurate Mechanism for Proportional Share Resource Allocation

    \vspace{0.2cm}
    Ion Stoica, Hussein Abdel-Wahab

    Old Dominion University 1996
  \end{shadowbox}

\end{frame}

\begin{frame}
  \frametitle{Предположения}

  \begin{itemize}
    \item Процессы конкурируют за общий разделяющийся по времени
      ресурс (CPU).
    \item Ресурс распределяется во временных квантах не больше чем
      $ q $.
    \item В начале каждого кванта времени для использования ресурса
      выбирается процесс.
    \item После того как процесс захватил ресурс он может
      использовать его весь квант времени либо освободить до
      окончания кванта времени.
    \item Каждому процессу назначается вес $ w_i $ определяющий
      относительную долю времени которая ему положена.
    \item Только процессы могут порождать временные кванты.
  \end{itemize}

\end{frame}

\begin{frame}
  \frametitle{Модель}
  \begin{columns}
    \column{0.5\textwidth}
    $$f_i(t) = \frac{w_i}{\sum_{j \in A(t)} w_j}$$

    $$S_i(t_0, t_1) = \int_{t_0}^{t_1} f_i(\tau) d\tau$$

    $$lag_i(t) = S_i(t_0, t) - s_i(t_0, t)$$

    \column{0.5\textwidth}
    $$V(t) = \int_{t_0}^{t} \frac{1}{\sum_{j \in A(\tau)} w_j} d\tau$$

    $$S_i(t_1, t_2) = w_i (V(t_2) - V(t_1))$$

    $$V(e) = V(t_0) + \frac{s_i(t_0, t)}{w_i}$$
  \end{columns}
\end{frame}

\begin{frame}
  \begin{columns}
    \column{0.5\textwidth}
    $$V(d) = V(e) + \frac{r}{w_i}$$

    $$ve^{(1)} = V(t_1)$$

    $$vd^{(k)} = ve^{(k)} + \frac{r^{(k)}}{w_i}$$

    \column{0.5\textwidth}
    $$ve^{(k+1)} = vd^{(k)}$$

    $$ve^{(k+1)} = ve^{(k)} + \frac{u^{(k)}}{w_i}$$

    $$V(t^+) = V(t) + \frac{lag_j(t)}{\sum_{i \in A(t^+)} w_i}$$

    $$V(t^+) = V(t) - \frac{lag_j(t)}{\sum_{i \in A(t^+)} w_i}$$

  \end{columns}
\end{frame}

\begin{frame}
  $$V(t^+) = V(t) + \frac{lag_j(t)}{\sum_{i \in A(t^+)} w_i} -
  \frac{lag_j(t)}{\sum_{i \in A(t^+)} w_i}$$

  $$-r_{max} < lag_k(t) < max(r_{max}, q)$$

  $$-q < lag_k(t) < q$$
\end{frame}

\begin{frame}
  \frametitle{Наивное сравнение}
  \begin{columns}
    \column{0.5\textwidth}
    \begin{figure}[ht]
      \centering
      \includegraphics[width=\linewidth]{res/EEVDF.png}
      \caption{EEVDF}
      \label{fig:my_label3}
    \end{figure}

    \column{0.5\textwidth}
    \begin{figure}[ht]
      \centering
      \includegraphics[width=\linewidth]{res/scx_nest.png}
      \caption{scx\_nest}
      \label{fig:my_label4}
    \end{figure}
  \end{columns}

  \begin{figure}[ht]
    \centering
    \includegraphics[width=0.7\linewidth]{res/EEVDF_sched_ext.png}
    \caption{EEVDF (sched\_ext)}
    \label{fig:my_label5}
  \end{figure}
\end{frame}

\begin{frame}
  \frametitle{Спасибо за внимание, вопросы?}

  \large Репозиторий с реализацией

  \begin{adjustwidth}{2cm}{0cm}
    \begin{thebibliography}{20}
      \bibitem{paper}
      \href{https://github.com/shgpavel/A1349}
      {https://github.com/shgpavel/A1349}
    \end{thebibliography}
  \end{adjustwidth}
\end{frame}

\begin{frame}
  \frametitle{Дополнительный слайд. Репозиторий scx}

  \large Репозиторий с другими sched\_ext планировщиками

  \begin{adjustwidth}{2cm}{0cm}
    \begin{thebibliography}{100}
      \bibitem{paper2}
      \href{https://github.com/sched-ext/scx}
      {https://github.com/sched-ext/scx}
    \end{thebibliography}
  \end{adjustwidth}
\end{frame}

\begin{frame}
  \frametitle{Дополнительный слайд. Обозначения 1}
  $ n $ -- количество процессов

  \vspace{0.03cm}
  $ n_a $ -- количество активных процессов

  \vspace{0.03cm}
  $ w_i $ -- вес $ i $-го процесса

  \vspace{0.03cm}
  $ q $ -- квант времени

  \vspace{0.03cm}
  $ m $ -- количество выделенных временных квантов

  \vspace{0.03cm}
  $ r $ -- запрошенное процессом время на исполнение

  \vspace{0.03cm}
  $ r_{max} $ -- максимально возможное время на исполнение

  \vspace{0.03cm}
  $ r_i^{(k)} $ -- продолжительность исполнения $ k $-го запроса $ i $-го
  процесса

  \vspace{0.03cm}
  $ u_i^{(k)} $ -- время которое фактически получает процесс $ i $ на
  k-ом запросе

  \vspace{0.03cm}
  $ t $ -- момент в реальном времени

  \vspace{0.03cm}
  $ t_a $ -- момент реального времени когда процесс $ i $ становится
  активным

  \vspace{0.01cm}
  $ t^{-} $ -- время прямо перед наступлением события

  \vspace{0.01cm}
  $ t^{+} $ -- время сразу после наступления события

  \vspace{0.01cm}
  $ A(t) $ -- множество всех активных в момент $ t $ процессов

  \vspace{0.01cm}
  $ W(t) $ -- сумма весов всех активных процессов

\end{frame}

\begin{frame}
  \frametitle{Дополнительный слайд. Обозначения 2}
  \vspace{0.03cm}
  $ f_{i}(t) $ -- доля процесса $ i $ во время $ t $

  \vspace{0.03cm}
  $ S_{i}(t_0, t_1) $ -- время обслуживания которое должен получить
  процесс $ i $ в идеальной системе

  \vspace{0.03cm}
  $ s_{i}(t_a, t) $ -- время обслуживания которое клиент $ i $ на
  самом деле получает

  \vspace{0.03cm}
  $ V(t) $ -- виртуальное время системы

  \vspace{0.03cm}
  $ e $ -- реальное подходящее время время запроса

  \vspace{0.03cm}
  $ d $ -- реальный дедлайн запроса

  \vspace{0.03cm}
  $ B $ -- множество активных процессов дедлайн которых в $ [e, d] $

  \vspace{0.03cm}
  $ C $ -- множество активных процессов дедлайн которых больше $ d $

  \vspace{0.03cm}
  $ ve_i^{(k)} $ -- виртуальное подходящее время $ k $-го запроса $ i $-го
  процесса

  \vspace{0.03cm}
  $ vd_i^{(k)} $ -- виртуальный дедлайн $ k $-го запроса $ i $-го
  процесса

  \vspace{0.03cm}
  $ lag_{i}(t) $ -- задержка обслуживания процесса $ i $ в момент
  времени $ t,\ (S - s) $

  \vspace{0.03cm}
  $ d_{neg} $ -- наибольший дедлайн среди всех процессов с отрицательными
  $ lag $ которые активны в момент времени $ t_1 $

\end{frame}

\end{document}
