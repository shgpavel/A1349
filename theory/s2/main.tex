\documentclass[12pt, oneside]{book}

\usepackage{graphicx}
\usepackage{amsmath}
\usepackage{amsfonts}
\usepackage{amssymb}
\usepackage{hyperref}

\usepackage[utf8]{inputenc}
\usepackage[T2A,T1]{fontenc}
\usepackage{ragged2e}

\usepackage{setspace}
\onehalfspacing
\setlength{\parindent}{1.25cm}
\everymath{\displaystyle}
\usepackage{tocloft}
\cftsetindents{section}{1em}{2em}
\renewcommand\cfttoctitlefont{\hfill\Large\bfseries}
\renewcommand\cftaftertoctitle{\hfill\mbox{}}
\setlength{\cftbeforesecskip}{0.1cm}
\setlength{\cftbeforesubsecskip}{0.1cm}
\setlength{\cftbeforetoctitleskip}{0cm}
\setlength{\cftaftertoctitleskip}{0.4cm}
\setcounter{tocdepth}{2}


\usepackage[english,russian]{babel}
\usepackage[a4paper, left=3cm, top=2cm, right=1.5cm, bottom=2cm]{geometry}

\def\letus{%
    \mathord{\setbox0=\hbox{$\exists$}%
             \hbox{\kern 0.125\wd0%
                   \vbox to \ht0{%
                      \hrule width 0.75\wd0%
                      \vfill%
                      \hrule width 0.75\wd0}%
                   \vrule height \ht0%
                   \kern 0.125\wd0}%
           }%
}

\makeatletter
\renewcommand{\@makeschapterhead}[1]{%
  \vspace*{0\p@}
  {\parindent \z@ \center
    \normalfont
    \LARGE \bfseries #1\par\nobreak
    \addcontentsline{toc}{chapter}{#1}
    \vskip 10\p@
}}
\makeatother

\makeatletter
\renewcommand{\chapter}{%
  \if@openright\cleardoublepage\else\clearpage\fi
  \thispagestyle{plain}
  \global\@topnum\z@
  \@afterindentfalse
  \secdef\@chapter\@schapter}

\renewcommand{\@makechapterhead}[1]{%
  \vspace*{0\p@}
  {\parindent \z@ \raggedright
    \normalfont
    \LARGE \bfseries \thechapter\quad #1\par\nobreak
    \vskip 20\p@
  }}
\makeatother

\makeatletter
\newcommand{\custompagestyle}{%
  \begingroup
  \renewcommand{\ps@plain}{%
    \renewcommand{\@oddfoot}{\hfil\thepage\hfil}
    \renewcommand{\@evenfoot}{\hfil\thepage\hfil}
    \renewcommand{\@oddhead}{}
    \renewcommand{\@evenhead}{}
  }%
  \pagestyle{plain}
}
\makeatother

\newenvironment{customquote}
  {\list{}{\leftmargin=0pt
          \rightmargin=0pt
          \topsep=0pt
          \partopsep=0pt
          \parsep=2pt
          \itemsep=0pt
        }%
   \normalfont\relax
  \item\relax}
  {\endlist}

\begin{document}
\thispagestyle{empty}
\begin{center}
\textbf{\large САНКТ-ПЕТЕРБУРГСКИЙ ГОСУДАРСТВЕННЫЙ УНИВЕРСИТЕТ} \\[2cm]

\text{\large ОТЧЕТ О НАУЧНО-ИССЛЕДОВАТЕЛЬСКОЙ РАБОТЕ} \\[2cm]

\text{\Large Шаго Павел Евгеньевич} \\[0.3cm]

\text{22.Б07-ПУ, 01.03.02 Прикладная математика и информатика} \\[1cm]

\textbf{\Large Планировщики 2} \\[3cm]

\text{\large Научный руководитель} \\[0.3cm]
\large{к.ф.-м.н., доцент Корхов В. В.} \\[10cm]
\text{Санкт-Петербург}\\ \today

\end{center}

\newpage
\custompagestyle
\setcounter{page}{2}
\renewcommand{\contentsname}{Содержание}
\tableofcontents

\newpage
\chapter*{Введение}
\begin{quote}
\quad Данная работа продолжает исследования связаные с планировщиками
и фокусируется на конкретных реализациях, архитектурных решениях, сравнительном
анализе.
\end{quote}


\chapter{Earliest eligible virtual deadline first (EEVDF)}
\section{Введение}
\begin{quote}
\bf EEVDF \normalfont --- планировщик потоков в ядре Linux,
который в 2023 году (с версии ядра Linux 6.6) вытеснил Completely
Fair Scheduler (CFS) в качестве планировщика по умолчанию.

CFS на тот момент использовался в ядре 16 лет, то есть ровно половину
всего времени существования ядра Linux. С работой CFS косвенно сталкивался любой
человек из-за широкого
Это решение поражает, казалось бы
CFS за такой срок уже доведен до идеала, поэтому причина замены 

Более полное базовое описание дано в главе 6 (раздел 3[6.3]) первого отчета о
научно-исследовательской работе[1].

В прошлом отчете данный планировщик потоков был упомянут,
но не рассмотрен 
в настоящей работе он рассматривается подробно.
\end{quote}
\section{Рассмотр причин отказа от CFS}

\chapter{sched\_ext}
\section{Введение}
\begin{quote}
  \bf sched\_ext \normalfont (scheduler extensions) --- инфраструктура ядра
  Linux позволяющая описывать планировщик набором BPF программ
  (поэтому sched\_ext также называют BPF scheduler).
  BPF (Berkeley Packet Filter) --- это такой специальный байт-код позволяющий 
  добавлять функциональность в ядро из пользовательского пространства, т.е. 
  исполнять код находящийся в непривилегированном пользовательском пространстве 
  в привилегированных режимах работы ядра. Изначально разработан для
  добавления функциональности сетевым картам, но с тех пор сильно поменялся,
  а конкретнее текущий BPF это синоним eBPF (extended BPF), изначальный же
  вариант сейчас называется cBPF (classic BPF) и практически не используется.
  BPF предоставляет ограниченный набор функций и спроектирован так, чтобы его
  можно было легко верфицировать, т.е. чтобы перед исполнением в ядре можно было
  легко и быстро узнать нет ли в данном коде небезопасных последовательностей
  инструкций. У BPF есть стандарт, хотя он достаточно условный и он может быть
  использован для доработки различных систем ядра в числе которых:
\end{quote}


\chapter*{Заключение}



\renewcommand{\bibname}{\large СПИСОК ИСПОЛЬЗОВАННЫХ ИСТОЧНИКОВ}
\begin{thebibliography}{9}
  \vspace*{0.3cm}
  \bibitem {1}
  Отчет о научно-исследовательской работе: Планировщики б. и. Павел Е. Шаго 2024
  \bibitem{2}
  Joseph Y-T. Leung\ Handbook of Scheduling: Algorithms, Models, and
  Performance Analysis CRC Press 2004
  \bibitem{3}
  Andrew S. Tanenbaum, Herbert Bos\ MODERN OPERATING SYSTEMS Pearson
  Education 2023
  \bibitem{4}
  Torvalds L. Linux Kernel [Электронный ресурс] / L. Torvalds.
  -- Режим доступа: \url{https://github.com/torvalds/linux} -- Дата обращения
  15.04.2025.
  \bibitem{5}
  sched-ext/scx [Электронный ресурс]. -- Режим доступа: \\
  \url{https://github.com/sched-ext/scx} -- Дата обращения 27.04.2025.
  \bibitem{6}
  lkml [Электронный ресурс]. -- Режим доступа: \url{https://lkml.org}
  -- Дата обращения 20.04.2025.
  \bibitem{7}
  Liz Rice Learning eBPF O’Reilly Media, Inc 2023
  \bibitem{8}
  William Stallings\ Operating Systems: Internals and
  Design Principles Pearson Education 2020
  \bibitem{9}
  Michael J. Morrison\ Resource Management and Scheduling
  in Multitasking Operating Systems CRC Press 2017
\end{thebibliography}

\end{document}
